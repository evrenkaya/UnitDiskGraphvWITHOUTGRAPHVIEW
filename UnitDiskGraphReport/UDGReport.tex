\documentclass{article}

% Used for wrapping table cell text
\usepackage{array}
\newcolumntype{L}{>{\centering\arraybackslash}m{3.2cm}}

\usepackage{longtable}
\usepackage{hyperref}
\hypersetup{
    colorlinks=true,     
    urlcolor=blue,
}

\newcommand{\dg}{G(n,r)}

\newtheorem{innercustomthm}{Theorem}

\newenvironment{customthm}[1]{\renewcommand\theinnercustomthm{#1}\innercustomthm}{\endinnercustomthm}

\setlength{\LTpost}{0pt}

\title{Algorithm Utility Tool Kit for Random Disk Graph Thresholds}
\author{Evren Kaya}
\date{July 17, 2015}

\begin{document}
\maketitle

\section{Introduction}
Wireless networks are usually modeled as disk graphs in the plane. Given a set $P$ of points in the plane and a positive parameter $r$, the {\em disk graph} is the geometric graph with vertex set $P$ which has a straight-line edge between two points $p,q\in P$ if and only if $|pq|\le r$, where $|pq|$ denotes the Euclidean distance between $p$ and $q$. If $r=1$, then the disk graph is referred to as {\em unit disk graph}.  
A {\em random geometric graph}, denoted by $\dg$, is a geometric graph formed by choosing $n$ points independently and uniformly at random in a unit square; two points are connected by a straight-line edge if and only if they are at Euclidean distance at most $r$, where $r=r(n)$ is a function of $n$ and $r \to 0$ as $n\to \infty$.

We say that two line segments in the plane {\em cross} each other if they have a point in common that is interior to both edges. Two line segments are {\em non-crossing} if they do not cross. Note that two non-crossing line segments may share an endpoint. A geometric graph is said to be {\em plane} if its edges do not cross, and {\em non-plane}, otherwise. A graph is {\em planar} if and only if it does not contain $K_5$ (the complete graph on 5 vertices) or $K_{3,3}$ (the complete bipartite graph on six vertices partitioned into two parts each of size $3$) as a minor. A {\em non-planar graph} is a graph which is not planar.

In order to test Theorems 1 and 4 from Biniaz et al. [1], we developed a computer program that serves as an algorithm utility kit for disk graphs. (An executable form of this program can be downloaded by clicking \href{https://github.com/evrenkaya/UnitDiskGraphvWITHOUTGRAPHVIEW}{here} (URL: $https://github.com/evrenkaya/UnitDiskGraphvWITHOUTGRAPHVIEW$) by clicking on ``UnitDiskGraph.jar'' and then ``View Raw'')

The following is a restatement of Theorems 1 and 4 from Biniaz et al. [1]:
\begin{customthm}{1}
\label{connected-k-thr}
Let $k\ge 2$ be an integer constant. Then, $n^{\frac{-k}{2k-2}}$ is a distance threshold function for $\dg$ to have a connected subgraph on $k$ points.
\end{customthm}

\begin{customthm}{4}
\label{plane-thr}
$n^{-\frac{2}{3}}$ is a threshold for $\dg$ to be plane.
\end{customthm}

In this paper, we explain some of the key implementation details of our program and then show our tests conducted for each theorem.

\newpage
\section{Implementation Details}
In this section, we describe the implementations of the key Java classes within our program.

\subsection*{Vertex}
A vertex is a point in the unit square. Every vertex has x,y coordinates as doubles, a boolean variable to store visited state(used in Breadth First Search), and an adjacency list represented as an ArrayList\textless Vertex\textgreater.

\subsection*{Edge}
A straight-line edge. Every edge has references to its endpoints, a boolean variable to keep track of whether it is intersecting with another edge, and a weight as a double.

\subsection*{UnitDiskGraph}
A random disk graph. Stores all vertices as an ArrayList\textless Vertex\textgreater, all edges as an ArrayList\textless Edge\textgreater, and the current distance threshold as a double. The following methods in this class are now described in more detail:\\ \\
createNewRandomVertices() - creates a new set of vertices uniformly at random in the unit square\\ \\
createNewConnectedEdges() - compares the Euclidean distance between every pair of vertices(Pythagoras' Theorem) and if two vertices are at most the distance threshold apart, then a new edge is created with these two vertices as its endpoints\\ \\
determineIntersectingEdges() - checks every pair of edges to see if they intersect at a point other than any common endpoints. This is done using Java's Line2D.linesIntersect() method\\ \\
determineSuperFreeEdges() - checks every pair of free edges to see if there are any other vertices inside of the super free edge rectangular region

\subsection*{BreadthFirstSearch}
A class representing the Breadth First Search algorithm. Contains a reference to the UnitDiskGraph object that it will be performing its search on. This class also stores all the connected components as an ArrayList\textless ArrayList\textless Vertex\textgreater\textgreater, in other words, a list of lists of vertices. The two methods of importance here are as follows:\\ \\
getConnectedComponentWith(Vertex startingVertex) - starts breadth first search at the given vertex and returns the entire connected component containing this vertex as an ArrayList\textless Vertex\textgreater \\ \\
determineAllConnectedComponents() - calls the above method for each vertex in the graph, keeping track of which connected components have been found already


\section{Testing}
In this section, we explain how we tested Theorems 1 and 4 using our algorithm utility program.

In order to test Theorem 1, we chose different values of $n$ and $k$ to input and then recorded the number of connected components with at least $k$ points that appeared above, below and at the distance threshold function using a small value $\varepsilon$. Theorem 1 is verified if there is \textbf{at least one} connected component on $k$ points \textbf{above} the distance threshold, and, if there are \textbf{no} connected components on $k$ points \textbf{below} the distance threshold. The following table shows our results for values of $n$ ranging from 50 - 10000.(Note: In this table, $a$ and $b$ are variables that replace $k$ and $2k-2$ respectively)

\begin{center}

\begin{longtable}{ | c | c | c | c | c | L | }
\hline
$n$ & $a=k$ & $b=2k-2$ & $\varepsilon$ & $r=n^{-(\frac{a}{b} + \varepsilon)}$ & \# Connected components with $\geq k$ points \\ 
\hline
50 & 5 & 8 & -0.05 & 0.10546 & 3 \\
50 & 5 & 8 & 0 & 0.08672 & 1 \\
50 & 5 & 8 & 0.05 & 0.07131 & 1 \\

50 & 10 & 18 & -0.05 & 0.13838 & 1 \\
50 & 10 & 18 & 0 & 0.11379 & 1 \\
50 & 10 & 18 & 0.05 & 0.09357 & 0 \\

50 & 30 & 58 & -0.05 & 0.16075 & 0 \\
50 & 30 & 58 & 0 & 0.13219 & 0 \\
50 & 30 & 58 & 0.05 & 0.10871 & 0 \\

& & & & & \\

100 & 5 & 8 & -0.05 & 0.07079 & 3 \\
100 & 5 & 8 & 0 & 0.05623 & 2 \\
100 & 5 & 8 & 0.05 & 0.04466 & 0 \\

100 & 10 & 18 & -0.05 & 0.09747 & 4 \\
100 & 10 & 18 & 0 & 0.07742 & 1 \\
100 & 10 & 18 & 0.05 & 0.06150 & 0 \\

100 & 30 & 58 & -0.05 & 0.11628 & 1 \\
100 & 30 & 58 & 0 & 0.09236 & 0 \\
100 & 30 & 58 & 0.05 & 0.07336 & 0 \\

& & & & & \\

500 & 5 & 8 & -0.05 & 0.02806 & 16 \\
500 & 5 & 8 & 0 & 0.02056 & 2 \\
500 & 5 & 8 & 0.05 & 0.01507 & 0 \\

500 & 10 & 18 & -0.05 & 0.04320 & 16 \\
500 & 10 & 18 & 0 & 0.03166 & 0 \\
500 & 10 & 18 & 0.05 & 0.02320 & 0 \\

500 & 30 & 58 & -0.05 & 0.05481 & 5 \\
500 & 30 & 58 & 0 & 0.04017 & 1 \\
500 & 30 & 58 & 0.05 & 0.02944 & 0 \\

& & & & & \\

1000 & 5 & 8 & -0.05 & 0.01883 & 24 \\
1000 & 5 & 8 & 0 & 0.01333 & 6 \\
1000 & 5 & 8 & 0.05 & 0.00944 & 1 \\

1000 & 10 & 18 & -0.05 & 0.03043 & 23 \\
1000 & 10 & 18 & 0 & 0.02154 & 5 \\
1000 & 10 & 18 & 0.05 & 0.01525 & 0 \\

1000 & 30 & 58 & -0.05 & 0.03965 & 2 \\
1000 & 30 & 58 & 0 & 0.02807 & 2 \\
1000 & 30 & 58 & 0.05 & 0.01987 & 0 \\

& & & & & \\

5000 & 5 & 8 & -0.05 & 0.00746 & 85 \\
5000 & 5 & 8 & 0 & 0.00487 & 9 \\
5000 & 5 & 8 & 0.05 & 0.00318 & 0 \\

5000 & 10 & 18 & -0.05 & 0.01348 & 144 \\
5000 & 10 & 18 & 0 & 0.00881 & 14 \\
5000 & 10 & 18 & 0.05 & 0.00575 & 0 \\

5000 & 30 & 58 & -0.05 & 0.01869 & 1 \\
5000 & 30 & 58 & 0 & 0.01221 & 8 \\
5000 & 30 & 58 & 0.05 & 0.00797 & 0 \\

& & & & & \\

10000 & 5 & 8 & -0.05 & 0.00501 & 130 \\
10000 & 5 & 8 & 0 & 0.00316 & 4 \\
10000 & 5 & 8 & 0.05 & 0.00199 & 0 \\

10000 & 10 & 18 & -0.05 & 0.00950 & 277 \\
10000 & 10 & 18 & 0 & 0.00599 & 21 \\
10000 & 10 & 18 & 0.05 & 0.00378 & 0 \\

10000 & 30 & 58 & -0.05 & 0.01352 & 3 \\
10000 & 30 & 58 & 0 & 0.00853 & 6 \\
10000 & 30 & 58 & 0.05 & 0.00538 & 0 \\

\hline
\end{longtable}
\begin{small}
\textbf{Table 1}: Theorem 1 testing data. $n$ is the number of points, $a$ and $b$ are variables that replace $k$ and $2k-2$ respectively, and $\varepsilon$ is a very small value used to vary the radius $r$. A random graph is generated for each row.
\end{small}
\end{center}

\newpage
In order to test Theorem 4, we set $a=2$ and $b=3$ as constants and only varied $n$ and $\varepsilon$. For each value of $n$, we recorded the number of intersecting edges within the disk graph that appeared above, below and at the distance threshold using $\varepsilon$. If there \textbf{exists} intersecting edges \textbf{above} the distance threshold, then the graph is not plane and this verifies Theorem 4. If there are \textbf{no} intersecting edges \textbf{below} the distance threshold, then the graph is plane and this also verifies Theorem 4. The following table shows our results for values of $n$ ranging from 50 - 10000.


\begin{center}
\begin{longtable}{ | c | c | c | L | }
\hline
$n$ & $\varepsilon$ & $r=n^{-(\frac{2}{3} + \varepsilon)}$ & \# Intersecting edges\\
\hline
50 & -0.001 & 0.07396 & 2\\
50 & 0 & 0.07368 & 2\\
50 & +0.001 & 0.07339 & 0\\

50 & -0.01 & 0.07662 & 0\\
50 & 0 & 0.07368 & 0\\
50 & +0.01 & 0.07085 & 0\\

50 & -0.05 & 0.08959 & 2\\
50 & 0 & 0.07368 & 2\\
50 & +0.05 & 0.06059 & 0\\

& & & \\

100 & -0.001 & 0.04663 & 0\\
100 & 0 & 0.04641 & 0\\
100 & +0.001 & 0.04620 & 0\\

100 & -0.01 & 0.04860 & 0\\
100 & 0 & 0.04641 & 0\\
100 & +0.01 & 0.04432 & 0\\

100 & -0.05 & 0.05843 & 0\\
100 & 0 & 0.04641 & 0\\
100 & +0.05 & 0.03686 & 0\\

& & & \\

500 & -0.001 & 0.01597 & 0\\
500 & 0 & 0.01587 & 0\\
500 & +0.001 & 0.01577 & 0\\

500 & -0.01 & 0.01689 & 0\\
500 & 0 & 0.01587 & 0\\
500 & +0.01 & 0.01491 & 2\\

500 & -0.05 & 0.02165 & 2\\
500 & 0 & 0.01587 & 0\\
500 & +0.05 & 0.01163 & 0\\

& & & \\

1000 & -0.001 & 0.01006 & 0\\
1000 & 0 & 0.00999 & 0\\
1000 & +0.001 & 0.00993 & 2\\

1000 & -0.01 & 0.01071 & 2\\
1000 & 0 & 0.00999 & 0\\
1000 & +0.01 & 0.00933 & 0\\

1000 & -0.05 & 0.01412 & 4\\
1000 & 0 & 0.00999 & 0\\
1000 & +0.05 & 0.00707 & 0\\

& & & \\

5000 & -0.001 & 0.00344 & 0\\
5000 & 0 & 0.00341 & 0\\
5000 & +0.001 & 0.00339 & 0\\

5000 & -0.01 & 0.00372 & 0\\
5000 & 0 & 0.00341 & 0\\
5000 & +0.01 & 0.00314 & 2\\

5000 & -0.05 & 0.00523 & 18\\
5000 & 0 & 0.00341 & 2\\
5000 & +0.05 & 0.00223 & 0\\

& & & \\

10000 & -0.001 & 0.00217 & 0\\
10000 & 0 & 0.00215 & 0\\
10000 & +0.001 & 0.00213 & 0\\

10000 & -0.01 & 0.00236 & 2\\
10000 & 0 & 0.00215 & 2\\
10000 & +0.01 & 0.00196 & 0\\

10000 & -0.05 & 0.00341 & 12\\
10000 & 0 & 0.00215 & 2\\
10000 & +0.05 & 0.00135 & 0\\

\hline
\end{longtable}
\begin{small}
\textbf{Table 2}: Theorem 4 testing data. A random graph is generated for each row. $a$ and $b$ are the constants 2 and 3 respectively.
\end{small} 
\end{center}

\section{Conclusion}
From Table 1, we see that for many combinations of $n$, $k$, and $\varepsilon$, there exists many connected components on $k$ points above the distance threshold
and none below the threshold, both verifying Theorem 1. From Table 2, we see that for high values of $n$($\geq 1000$), there exists intersecting edges above the distance threshold and that there are no intersecting edges below the threshold. This verifies that $n^{-\frac{2}{3}}$ is a distance threshold for $\dg$ to be plane(i.e. Theorem 4).
\\
\\
\begin{small}Acknowledgments: I would like to thank both of my supervisors, Professor Anil Maheshwari and PhD student Ahmad Biniaz for assisting me in creating this report.\end{small}

\bibliographystyle{abbrv}
\bibliography{biblio}

\end{document}
